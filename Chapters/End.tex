\chapter{Závěr}
\label{sec:End}
\

Cílem této bakalářské práce bylo určit optimální rychlost jízdy automobilu v~různých
fázích jeho pohybu.

V~práci byly popsány metody řízení automobilu, zahrnující detekci dráhy a~řízení
motorů včetně servomotorů na základě získaných dat o~trase. Součástí práce bylo také
porovnání s~manuálním řízením a~metody přenosu dat v~reálném čase. Značná část práce
byla věnována senzorům – jejich konfiguraci, výběru a~filtraci signálu.

K~detekci dráhy během jízdy byly využity tři kroky předzpracování obrazu: filtrování
mediánem, normalizace obrazu a~prahování pomocí průměrné hodnoty. Pro lepší kontrolu
servomotoru byl implementován PID regulátor a~rychlost PWM motoru byla optimalizována
na základě dat získaných z~kamery a~senzorů.

Analýza dat ze senzorů ukázala, že magnetometr nelze používat v~blízkosti motorů a~je
nezbytné filtrovat signál z~akcelerometru. Při filtraci signálu byly nejlepší výsledky 
dosaženy pomocí filtrů \uv{Jednopólová dolní propust} a~\uv{Čtyřstupňová dolní 
propust}. První zmíněný byl zvolen pro finální řešení kvůli lepší redukce šumu.

Pro optimalizaci rychlosti na základě senzorů byl použit výpočet optimální boční
rychlosti pro~plynulou jízdu zatáčkou bez smyku.

Pro srovnání automatického se senzory i~bez a~manuálního řízení byl navržen 
a~implementován algoritmus prahování k~počítání kol na základě dat z~gyroskopu.

Implementace a~porovnání ukázaly, že řízení na základě vybraných senzorů je rychlejší
oproti manuálnímu i~automatickému řízení bez senzorů.

\endinput