\chapter{Optimalizace rychlosti}
\label{sec:SpeedOptimization}
\

Abychom optimalizovali rychlost je nutné správně rozlišovat různé fáze jízdy. Na
dráze, která je popsaná v~kapitole \ref{sec:PlatformControl}, jsou dvě fáze:
\begin{itemize}
	\item{rovina;}
	\item{zatáčka.}
\end{itemize}

Rozlišujeme to na základě boční rychlosti, tzn. pokud boční rychlost $a$ je menši než
optimální, jedná se o jízdě po rovinně, v opačném případě bude jízda v~zatáčce.

V zatáčce musíme spočítat takové zpomalení, aby boční rychlost byla
co nejblíž ke koeficientu. Můžeme to udělat na základě formule optimální boční
rychlosti:
\begin{equation}
v = \sqrt{a_{max} * r},
\end{equation}
kde $v$ je optimální zpomalení, $a_{max}$ je maximální boční rychlost a r je poloměr
zatáčky. Po experimentech pro $a_{max}$ je zvolena hodnota 0,2. Tuto hodnotu
optimálního zpomalení používáme i pro výpočet zrychlení na rovině. Poloměr zatáčky můžeme
spočítat na základě uhlové rychlosti:
\begin{equation}
r = \frac{v}{\omega},
\end{equation}
kde $v$ je rychlost auta a $w$ je uhlová rychlost.

Hodnoty pro boční zrychlení a uhlovou rychlost dostaneme z $A_y$ a $\Omega_z$. $A_y$ je 
zobrazena na obrázku \ref{fig:FXOS_Orientation}. $\Omega_z$ je zobrazena na obrázku
\ref{fig:FXAS_Orientation}.
Tyto hodnoty před použitím je potřeba zkonvertovat ze surových
hodnot.

Pro konvertování $A_y$ byl použit vzorec:
\begin{equation}
a = \frac{A_y * k * g}{x_{max}},
\end{equation}
kde $a$ je boční zrachlení, $A_y$ je surová hodnota akcelerometru ose y, 
$k$ je rozsah akcelerometru,  $g$ je konstanta gravitačního zrychlení a $x_{max}$ je  
maximální 14bitová surová hodnota.  Pro~konfigurace akcelerometru, která je popsaná v kapitole
\ref{sec:Sensors}, byl zvolen koeficient $k$ = 4, $g$ = 10 a $a_{max}$ je 8191.

Pro konvertování $\Omega_z$ byl použit vzorec:
\begin{equation}
\omega = \Omega_z * k * \frac{\pi}{180},
\end{equation}
kde $\omega$ je uhlová rychlost, $\Omega_z$ je surová hodnota gyroskopu ose z 
a $k$ je citlivost senzoru ve stupních za sekundu.
Pro konfiguraci gyroskopu, která je popsaná v kapitole
\ref{sec:Sensors}, byla použita hodnota k = 0,03125.

\endinput
