\chapter{Optimalizace rychlosti}
\label{sec:SpeedOptimization}
\vspace{-20pt}
\

Pro optimalizace rychlosti potřebně rozlišovat různé fáze jízdy.
Na draze, která je popsaná v kapitole \ref{sec:PlatformControl},
je dvě fáze, rovina a zatáčka. To můžeme rozlišovat na základě
boční rychlosti, tzn pokud boční rychlost je menši než optimální boční rychlost,
jízda po rovinně, v jiném případě jízda v zatáčce.

V zatáčce musíme spočítat takové zpomalení, aby boční rychlost byla co nejblíž
ke koeficientu. To můžeme udělat na základě formule optimální boční rychlosti:
\begin{equation}
v = \sqrt{a_{max} * r},
\end{equation}
kde $v$ je optimální zpomalení, $a_{max}$ je maximální boční rychlost a r je
poloměr zatáčky. Po experimentech pro $a_{max}$ je zvolena hodnota 0,2. Poloměr zatáčky můžeme spočítat na základě bočního zrychlení:
\begin{equation}
r = \frac{v}{w},
\end{equation}
kde $v$ je rychlost auta a $w$ je boční zrychlení.

Hodnoty pro boční rychlost a boční zrychlení dostaneme z $y$ ose akcelerometru
a $z$ ose gyroskopu. Tyto hodnoty před použitím je potřeba zkonvertovat z
surových hodnot.

Pro konvertování dat z akcelerometru byla použita formule:
\begin{equation}
a = \frac{x * k}{x_{max}},
\end{equation}
kde $x$ je surová hodnota, $k$ je rozsah akcelerometru a $x_{max}$ je maximální surová hodnota. Pro konfigurace akcelerometru, která je popsaná v kapitole \ref{sec:Sensors},
k = 4 a maximální hodnota čísla 14 bit je 8191.

Pro konvertování dat z gyroskopu byla použita formule:
\begin{equation}
w = x * k * \frac{\pi}{180},
\end{equation}
kde $w$ je boční zrychlení, $k$ je citlivost senzoru v stupně za sekundu a $x$
je surová hodnota. Pro konfigurace gyroskopu, která je popsaná
v kapitole \ref{sec:Sensors}, k = 0,03125.

\endinput
