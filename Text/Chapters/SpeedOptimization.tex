\chapter{Optimalizace rychlosti}
\label{sec:SpeedOptimization}
\

Abychom optimalizovali rychlost je nutné správně rozlišovat různé fáze jízdy. Na
dráze, která je popsaná v~kapitole \ref{sec:PlatformControl}, jsou dvě fáze:
\begin{itemize}
	\item{rovina;}
	\item{zatáčka.}
\end{itemize}

Rozlišujeme to na základě boční rychlosti, tzn. pokud boční rychlost je menši než
optimální, jedná se o jízdě po rovinně, v opačném případě bude jízda v~zatáčce.

V zatáčce musíme spočítat takové zpomalení nebo zrychlení, aby boční rychlost byla
co nejblíž ke koeficientu. Můžeme to udělat na základě formule optimální boční
rychlosti:
\begin{equation}
v = \sqrt{a_{max} * r},
\end{equation}
kde $v$ je optimální zpomalení, $a_{max}$ je maximální boční rychlost a r je poloměr
zatáčky. Po experimentech pro $a_{max}$ je zvolena hodnota 0,2. Tuto hodnotu
optimálního zpomalení používáme i pro výpočet zrychlení. Poloměr zatáčky můžeme
spočítat na základě bočního zrychlení:
\begin{equation}
r = \frac{v}{w},
\end{equation}
kde $v$ je rychlost auta a $w$ je boční zrychlení.

Hodnoty pro boční rychlost a boční zrychlení dostaneme z $y$ osy akcelerometru a $z$
osy gyroskopu. Tyto hodnoty před použitím je potřeba zkonvertovat ze surových
hodnot.

Pro konvertování dat z akcelerometru byl použit vzorec:
\begin{equation}
a = \frac{x * k}{x_{max}},
\end{equation}
kde $x$ je surová hodnota, $k$ je rozsah akcelerometru a $x_{max}$ je maximální
surová hodnota. Pro~konfigurace akcelerometru, která je popsaná v kapitole
\ref{sec:Sensors}, byl zvolen koeficient k = 4 a maximální surová hodnota je 8191
(14 bit).

Pro konvertování dat z gyroskopu byl použit vzorec:
\begin{equation}
w = x * k * \frac{\pi}{180},
\end{equation}
kde $w$ je boční zrychlení, $k$ je citlivost senzoru v stupně za sekundu a $x$ je
surová hodnota. Pro konfigurace gyroskopu, která je popsaná v kapitole
\ref{sec:Sensors}, byla použita hodnota k = 0,03125.

\endinput
