\chapter{Testování}
\label{sec:Testing}
\vspace{-20pt}
\

V této kapitole je popsáno testování algoritmu optimalizace rychlosti na dráze, 
která je popsaná v kapitole~\ref{sec:PlatformControl}, a porovnána s manuálním
řízením. 

Pro porovnání potřebujeme spočítat za kolik ticku auto projelo jedno kolo. Tick je jednotka času, která představuje jeden cyklus časovače.
Na to můžeme využít data bočního zrychlení, která jsou zobrazena na obrázku \ref{fig:Laps}.
Levý obrázek je zrychlení automatického řízení a pravý obrázek je zrychlení 
manuálního řízení.

\begin{figure}[!h]
    \centering
    \subfloat[Automatické řízení]{\includegraphics[width = .5\textwidth]{Figures/LapAuto.png}\label{fig:LapsAuto}}
    \subfloat[Manuální řízení]{\includegraphics[width = .5\textwidth]{Figures/LapManual.png}\label{fig:LapsManual}}
    \caption{Boční zrychlení během experimentů.}
    \label{fig:Laps}
\end{figure}

Implementace algoritmu prahování zajišťuje pro každý datový bod hodnotu
úhlové rychlosti z gyroskopu na ose z. Pokud tato hodnota překročí prahovou hodnotu, a 
zároveň nebyl ještě zaznamenán polokruh, algoritmus zaznamená časovou známku detekce jako 
počátek polokruhu. Pokud hodnota úhlové rychlosti klesne pod prahovou hodnotu 
a byl již zaznamenán polokruh, znamená to, že byl detekován úplný kruh. 
Algoritmus poté zvýší počet detekovaných kol o jednu.
Tento proces se opakuje pro každý datový bod, čímž algoritmus umožňuje sledování a 
počítání kol na základě úhlové rychlosti zaznamenané gyroskopem.
Použitý algoritmus je ve výpisu:
\begin{lstlisting}[
	caption=Počet kol,
	label=lst:countLap
]
threshold = 3000
half_lap_detected = False
lap_count = 0
laps = []

for data_point in data:
    if data_point.sensor.gyro.z >= threshold and not half_lap_detected:
        laps.append((lap_count, data_point.timestamp))
        half_lap_detected = True
    elif data_point.sensor.gyro.z <= -threshold and half_lap_detected:
        lap_count += 1
        half_lap_detected = False
\end{lstlisting}

Porovnání automatického a manuálního řízení je v tabulce \ref{tab:Comparison}.
\begin{table}[!h]
    \centering
    \begin{tabular}{cccc}
        \hline
        \textbf{Metoda řízení} & \textbf{1 Kolo} & \textbf{2 Kolo} & \textbf{3 Kolo} \\
        \hline
        Automatické           & 942       & 562 & 343          \\
        Manuální 			  & 1178       & 1085 & 1161           \\
        \hline
    \end{tabular}
    \caption{Porovnání manuálního a automatického řízení.}
    \label{tab:Comparison}
\end{table}

Na základě tabulky můžeme vyvodit závěr, že automatické řízení
je mnohem rychlejší než manuální.
\endinput
