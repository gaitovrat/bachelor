\chapter{Testování}
\label{sec:Testing}
\

V této kapitole je popsáno testování algoritmu optimalizace rychlosti na dráze,
která je popsaná v kapitole~\ref{sec:PlatformControl}, a porovnána s manuálním
řízením i řízením bez optimalizace na základe senzoru.

Pro porovnání potřebujeme spočítat za kolik tiku auto projelo jedno kolo. Na to můžeme využít data
úhlové rychlosti, která je zobrazena na obrázku \ref{fig:Laps}.

\begin{figure}[!h]
    \centering
    \subfloat{\includegraphics[width = .5\textwidth]{Figures/LapAuto.png}\label{fig:LapsAuto}}
    \subfloat{\includegraphics[width = .5\textwidth]{Figures/LapAutoNoSensors.png}\label{fig:LapsAuto}} \\
    \subfloat{\includegraphics[width = .5\textwidth]{Figures/LapManual.png}\label{fig:LapsManual}}
    \captionsetup{justification=centering}
    \caption{Boční zrychlení během experimentů. Vlevo nahoře: Automatické řízení se senzory. Vpravo nahoře: Automatické řízení bez senzorů. Dolů: Manuální řízení.}
    \label{fig:Laps}
\end{figure}

Implementace algoritmu prahování zajišťuje pro každý datový bod hodnotu úhlové
rychlosti z~gyroskopu na ose z. Pokud tato hodnota překročí prahovou hodnotu, a
zároveň nebyl ještě zaznamenán polokruh, algoritmus zaznamená časovou známku detekce
jako počátek polokruhu. Polokruh je stav, kdy gyroskop má nejvyšší hodnotu, což z pohledu drahý je jedna ze dvou zatáček. Pokud hodnota úhlové rychlosti klesne pod prahovou hodnotu a
byl již zaznamenán polokruh, znamená to, že byl detekován úplný kruh. Algoritmus
poté zvýší počet detekovaných kol o jednu. Tento proces se opakuje pro každý datový
bod, čímž algoritmus umožňuje sledování a počítání kol na základě úhlové rychlosti
zaznamenané gyroskopem. Použitý algoritmus je ve výpisu:
\begin{lstlisting}[language = python, caption = Počet kol, label = lst:countLap]
threshold = 3000
half_lap_detected = False
lap_count = 0
laps = []

for data_point in data:
    if data_point.sensor.gyro.z >= threshold and not half_lap_detected:
        laps.append((lap_count, data_point.timestamp))
        half_lap_detected = True
    elif data_point.sensor.gyro.z <= -threshold and half_lap_detected:
        lap_count += 1
        half_lap_detected = False
\end{lstlisting}

Porovnání automatického s senzory i bez senzoru a manuálního řízení je v tabulce \ref{tab:Comparison}.
\begin{table}[!h]
    \centering
    \begin{tabular}{cccc}
        \hline
        \textbf{Řízení} & \textbf{1 Kolo} & \textbf{2 Kolo} & \textbf{3 Kolo} \\
        \hline
        Automatické s senzory          & 942       & 562 & 343          \\
        Automatické bez senzoru & 1196 & 1175 & 1138 \\
        Manuální 			  & 1178       & 1085 & 1161           \\
        \hline
    \end{tabular}
    \caption{Porovnání manuálního a automatického řízení.}
    \label{tab:Comparison}
\end{table}


Na základě tabulky můžeme vyvodit závěr, že automatické řízení bez senzorů je pomalejší než manuální řízení, a zároveň, že automatické řízení se senzory je nejrychlejší.

\endinput
