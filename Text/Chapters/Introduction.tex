\chapter*{Úvod}
\label{sec:Introduction}
\addcontentsline{toc}{chapter}{\protect\numberline{}Úvod}\

Autonomní řízení automobilů je v současné době velmi populárním tématem, přičemž
výrobci automobilů se předhánějí o prvenství v této oblasti. Jedná se především o
velmi komplexní oblast, kde musí být správně vyřešeny všechny možné situace, jelikož
jakýkoliv přehlédnutý detail může způsobit újmu na zdraví. Z tohoto důvodu je
autonomní řízení částí populace stále vnímáno jako něco  nepřijatelného, ovšem
zvyšující se rychlost vývoje naznačuje, že v následujících letech dojde k~významnému
pokroku v této oblasti a setkávání autonomně řízenými auty bude běžným jevem.

V této bakalářské práci bude popsán způsob automatického řízení auta a korekce rychlosti na základě informací z pohybových senzorů. Práce začne popisem modelu automobilu pro testování. Následně bude popsaný současný stav problematiky automatického řízení a pohybových senzorů. Třetí část práce bude věnována algoritmům pro automatické a manuální řízení. Další část se bude zabývat metodami pro zobrazování a ukládání hodnot auta během jízdy. Poté budou otestovány a vybrány vhodné filtry a senzory pro detekci fáze jízdy. V závěrečné části práce bude popsán algoritmus pro optimalizaci rychlosti a provedeno porovnání s manuálním řízením.

\endinput
