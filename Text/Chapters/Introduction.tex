\chapter*{Úvod}
\label{sec:Introduction}
\addcontentsline{toc}{chapter}{\protect\numberline{}Úvod}\

Autonomní řízení automobilů je v současné době velmi populárním tématem, přičemž výrobci automobilů
se předhánějí o prvenství v této oblasti. Jedná se především o velmi komplexní oblast, kde
musí být správně vyřešeny všechny možné situace, jelikož jakýkoliv přehlédnutý detail může
způsobit újmu na zdraví. Z tohoto důvodu je autonomní řízení částí populace stále vnímáno jako něco 
nepřijatelného, ovšem zvyšující se rychlost vývoje naznačuje, že v následujících letech dojde k významnému pokroku v této oblasti a setkávání autonomně řízenými auty bude běžným jevem.

V této práci bude popsán způsob automatického řízení auto a korigovaní rychlosti na základe informace z pohybových senzorů.
Pro tento účel v první častí práce začne popsáním modelu automobilu pro testování. Druhá část poté popisuje současný stav problematiky automatického řízení a pohybových senzoru. V třetí části práce se popsán algoritmus automatického a manuálního řízení. V čtvrté části popsán způsob zobrazování a ukládání hodnot auta během jízdy. V páté častí otestovány a vybraný vhodné senzory pro zjištění fázi auta. Následné v 6 častí je otestovaný vhodné filtry pro snazší detekce fáze. V konce práce je popsán algoritmus optimalizace rychlosti a je porovnán s manuálním řízením.

\endinput
