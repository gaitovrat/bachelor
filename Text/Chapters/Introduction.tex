\chapter*{Úvod}
\label{sec:Introduction}
\addcontentsline{toc}{chapter}{\protect\numberline{}Úvod}\

Cílem této bakalářské práce je vytvořit spolehlivou řízení motoru na modelu
automobilu používaném na soutěži NXP Cup.

Autonomní řízení automobilů je v dnešní době velmi populárním tématem a výrobci automobilů
se předhánějí o prvenství v této oblasti. Jedná se ovšem o velmi komplexní oblast, kde
musí být perfektně vyřešeny všechny možné situace, jelikož jakýkoliv přehlédnutý detail může
skončit i smrtí cestujících.

Z tohoto důvodu je autonomní řízení částí populace stále vnímáno jako něco nepřijatelného,
ovšem rychlost vývoje naznačuje, že během několika let dojde k výraznému rozšíření a potkávání
autonomně řízených aut bude na denním pořádku.

V této práci budou otestovány vhodné senzory pro zjištění fázi auta.
Pak podle fáze motor modelu automobilu bude si regulovat.
Bude zde popsán i způsob přenosu a ukládání naměřených dat.
Následně budou naměřená data zpracována na počítači.
Pro snazší detekci fáze, auto bude otestováno a porovnáno několik filtrů
pro redukci šumu aplikovaných na signál naměřený těmito senzory.
Budou také porovnané automatické a manuální řízení.
Následně budou vybrané filtry implementovány k použití na mikropočítači použitém v modelu automobilu.
\endinput
