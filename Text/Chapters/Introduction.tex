\chapter*{Úvod}
\label{sec:Introduction}
\addcontentsline{toc}{chapter}{\protect\numberline{}Úvod}\

Autonomní řízení automobilů je v současné době velmi populárním tématem, 
přičemž výrobci automobilů se předhánějí o prvenství v této oblasti. 
Jedná se především o velmi komplexní oblast, kde musí být správně 
vyřešeny všechny možné situace, jelikož jakýkoliv přehlédnutý detail může
způsobit újmu na zdraví. Z tohoto důvodu je autonomní řízení částí 
populace stále vnímáno jako něco  nepřijatelného, ovšem zvyšující se 
rychlost vývoje naznačuje, že v následujících letech dojde k~významnému 
pokroku v této oblasti a setkávání autonomně řízenými auty bude běžným 
jevem.

V této práci byl popsán způsob automatického řízení auta a korigovaní 
rychlosti na základe informace z pohybových senzorů. Pro tento účel 
práce se začíná popsáním modelu automobilu pro~testování. Poté byl
popsán současný stav problematiky automatického řízení a pohybových 
senzoru. Třetí část práce popisuje algoritmus automatického a manuálního 
řízení. Následující část se věnuje způsobu zobrazování a ukládání hodnot 
auta během jízdy. Následné byly otestovaný a vybraný vhodné filtry a 
senzory pro detekci fázi auta. V konce práce je popsán algoritmus 
optimalizace rychlosti a je porovnán s manuálním řízením.

\endinput
